% mn2esample.tex
%
% v2.1 released 22nd May 2002 (G. Hutton)
%
% The mnsample.tex file has been amended to highlight
% the proper use of LaTeX2e code with the class file
% and using natbib cross-referencing. These changes
% do not reflect the original paper by A. V. Raveendran.
%
% Previous versions of this sample document were
% compatible with the LaTeX 2.09 style file mn.sty
% v1.2 released 5th September 1994 (M. Reed)
% v1.1 released 18th July 1994
% v1.0 released 28th January 1994

\documentclass[useAMS,usenatbib]{mn2e}
\usepackage{amsmath,graphicx,xspace,hyperref,url,acronym,aas_macros,amssymb}


% If your system does not have the AMS fonts version 2.0 installed, then
% remove the useAMS option.
%
% useAMS allows you to obtain upright Greek characters.
% e.g. \umu, \upi etc.  See the section on "Upright Greek characters" in
% this guide for further information.
%
% If you are using AMS 2.0 fonts, bold math letters/symbols are available
% at a larger range of sizes for NFSS release 1 and 2 (using \boldmath or
% preferably \bmath).
%
% The usenatbib command allows the use of Patrick Daly's natbib.sty for
% cross-referencing.
%
% If you wish to typeset the paper in Times font (if you do not have the
% PostScript Type 1 Computer Modern fonts you will need to do this to get
% smoother fonts in a PDF file) then uncomment the next line
% \usepackage{Times}

%%%%% AUTHORS - PLACE YOUR OWN MACROS HERE %%%%%
% Macros
\newcommand{\mgii}{Mg~{\sc ii}\xspace}
\newcommand{\feii}{Fe~{\sc ii}\xspace}

\newcommand{\ntcomment}[1]{\texttt{\\ \noindent[NT: #1]}}


% Acronyms
\acrodef{los}[LOS]{line-of-sight}
\acrodef{lss}[LSS]{large scale structure}
\acrodef{sdss}[SDSS]{Sloan Digital Sky Survey}
\acrodef{redmapper}[redMaPPer]{red-sequence Matched-filter Probabilistic Percolation}
\acrodef{qso}[QSO]{quasi-stellar object}
\acrodef{snr}[SNR]{signal-to-noise ratio}
%%%%%%%%%%%%%%%%%%%%%%%%%%%%%%%%%%%%%%%%%%%%%%%%

\title[The incidence of \mgii\ as a function of cluster masses]{Cold
  gas within and around galaxy clusters: The incidence of \mgii\ as a
  function of cluster masses \ntcomment{better title TBD}}
\author[Crighton, Tejos \& L\'opez]{
\parbox[t]{\textwidth}{
\vspace{-1.0cm} 
Neil H. Crighton$^{1}$\thanks{E-mail: \href{mailto:neilcrighton@gmail.com}{neilcrighton@gmail.com}},
Nicolas Tejos$^{2}$ and
Sebasti\'an L\'opez$^{3}$
}
\vspace*{6pt} \\ 
$^{1}$ Centre for Astrophysics and Supercomputing, Swinburne University of Technology, Hawthorn, Victoria 3122, Australia\\
$^{2}$ Department of Astronomy and Astrophysics, UCO/Lick Observatory, University of California, 1156 High Street, Santa Cruz, CA 95064, USA\\
$^{3}$ Departamento de Astronom\'ia, Universidad de Chile, Casilla 36-D, Santiago, Chile\\
\vspace*{-0.5cm}}

\begin{document}

\date{Draft v1.0}

\pagerange{\pageref{firstpage}--\pageref{lastpage}} \pubyear{2014}

\maketitle

\label{firstpage}

\begin{abstract}
We investigate the incidence of \mgii\ absorption line systems in
galaxy clusters as a function of cluster mass. Our results are....
\end{abstract}

\begin{keywords}
--intergalactic medium --quasars: absorption lines --galaxies: clusters:
intracluster medium --large scale structure of the Universe --galaxies:
evolution
\end{keywords}

\section{Introduction}
\acresetall
\ntcomment{brief introduction}


\section{Data}\label{sec:data}

\subsection{Galaxy clusters}\label{sec:data:clusters}

We use data from the \ac{redmapper} catalog published by
\citet{Rykoff2014} from SDSS DR8 data (refs). The \ac{redmapper}
catalog is mainly based on optical photometry through the red sequence
technique (refs).

The catalog comes from a fully automatized algorithm briefly described
as follows:

(i) 


\ntcomment{\\
- Description of the \ac{redmapper} cluster catalog\\
- Explain why \ac{redmapper}: advantages\\
- Clusters centers as probabilities\\
- Cluster members as probabilities\\
- Cluster Richness corrected by incompleteness\\
- Photometric uncertainties are small\\
- Some spectroscopic redshift exists\\
- Adopted richness--mass relationship\\
- Explain completeness and purity\\}


\subsection{\mgii\ absorption line systems}\label{sec:data:mgii}

We used data from the JHU-SDSS metal absorption line catalog published
by \citet{Zhu2013} from SDSS DR7 \ac{qso} data \citep{Schneider2010,
  Hewett2010}. This catalog provides redshift, rest-frame equivalent
widths and velocity dispersion (among other quantities) of XXX
\mgii\ $\lambda\lambda 2796, 2803$ systems found in the spectra of XXX
\acp{qso} from the \ac{sdss} DR7 and DR9. The catalog also gives the
same information for the corresponding \mgii $\lambda 2853$ and \feii
$\lambda 2344$, $\lambda 2383$, $\lambda 2586$, and $\lambda 2600$. The
catalog comes from a fully automatized algorithm briefly described as
follows:

\begin{itemize}
\item (i) Each QSO continuum is fitted by a set of $12$ non-negative
eigen-spectra following the nonnegative matrix factorization technique
\citep[NMF;][]{Lee1999;Blanton2007}, and normalized accordingly.

\item (ii) Median filters of $\sim 8$ and $\sim 4$ times the expected
\mgii~spectral extent is applied to the resulting normalized spectrum
in step (i), in order remove intermediate and small scales
fluctuations, respectively. Each QSO continuum is re-normalized
accordingly.

\item (iii) For each normalized QSO, a search window for \mgii
  absorption lines is defined between redshifts $(1+z_{\rm
    QSO})\frac{\lambda_{\rm CIV}}{\lambda_{\rm MgII}} - 0.08 \le z \le
  z_{\rm QSO} - 0.04$ (i.e. between the \ion{C}{4} and the \ion{Mg}{2}
  broad emission lines), excluding spectral zones associated with sky
  emission lines (e.g. \ion{O}{1} and OH) and Galactic absorption
  (e.g. \ion{Ca}{2} and Na~D).

\item (iv) \mgii candidates are looked for by fitting a multi-line
  model of the \mgii doublet plus four strong \ion{Fe}{2} lines
  ($\lambda 2344$,$\lambda 2383$,$\lambda 2586$,$\lambda 2600$). Still
  the presence of \ion{Fe}{2} is not a requirement.

\item (v) For each of these candidates, two Gaussian to each member of
  the \mgii doublet are fitted (allowing them to have different
  amplitudes and positions), and those candidates whose doublet
  separations are off by $1 \AA$ from the expectation, are discarded.

\item (vi) The remaining candidates are stored and ranked according to
  signal-to-noise criteria based for the \mgii and \ion{Fe}{2}
  lines. Higher priority is given to candidates showing as \mgii {\it
    and} all four \ion{Fe}{2} lines at $SNR(FeII)>2$, than those
  showing only as \mgii.\footnote{We note that regions around the QSO
    \ion{C}{3} emission line are treated slightly different
    \citep[see][for further details]{Zhu2013}.}

\item (vii) A final list of `robust' \mgii systems is given by those
  satisfying $SNR(MgII,2796)>4$, $SNR(MgII,2803)>2$, and their
  properties measured and stored.

\end{itemize}


We chose this catalog mostly because it is the largest currently
available, and that it comes the SDSS footprint where the cluster
catalog used in this paper also comes from \ref{sec:data:clusters}. We
also emphasize that the JHU-SDSS has a very well understood selection
function. The catalog is about $\lesssim XX\%$ complete at $W_r \gtrsim
2$~\AA, and $\lesssim 70\%$ at $W_r \gtrsim 1$~\AA (see their figure 7
and 9).

In the present paper, we will focus mostly on \mgii systems having $W_r
\gtrsim 1$~\AA at redshifts $XX\le z \le XX$, for a total of $XXX$
systems. Their redshift distribution is shown as the .... line in
\ref{fig:fig1}. The completeness level in this case is not
significantly different than the overall catalog (see their figure
7). \ntcomment{Are we going to use those below below CIV emission?; if
  so, say something about it (higher contamination rate).} In any case,
the completeness level is not an issue to our analysis because .... ,

The main observables for our present work are redshift ($z$),
equivalent width of the $\lambda2796$ line ($W_r^{2796}$), and velocity
dispersion ($\sigma_v^{2796}$). We note that the majority of the \mgii
lines in the JHU-SDSS catalog are saturated making $W_r^{2796}$ and
$\sigma_v^{2796}$ correlated (see their figure 6).


\section{Data analysis and results}
\subsection{Cross-match between clusters and \mgii\ systems}
\ntcomment{\\
- Show plots of redshift overlap\\
- Explain subsample of the cluster catalog used in this analysis\\
- Explain subsample of the \mgii\ catalog used in this analysis\\
- Justify the maximum scale adopted (~40 Mpc? use as motivation the mean distance between clusters?)
- Define redshift path per cluster\\
- Define hits\\
}
\begin{figure}
 \begin{minipage}{0.5\textwidth}
    \centering
    \includegraphics[width=5cm]{figs/figure_sample.pdf}
  \end{minipage}

\caption{Clusters and \mgii\ redshift distribution...}
\label{fig:fig1}

\end{figure}


\subsection{The incidence \mgii\ in clusters}
\ntcomment{\\
- Define incidence in clusters, $dN/dz$\\
- Define incidence as a function of co-moving separation\\
- Define incidence as a function of virial radii; emphasize why is important to scale by virial mass: massive clusters are larger. Dark matter haloes are expected to show self-similarity, others)\\
- Plot these two quantities for our full samples in both co-moving and r200 separations.\\
- Mention whether these two seem consistent with each other but leave interpretation to the Discussion section. If no qualitative differences, stick to r200.}

\begin{figure}
 \begin{minipage}{0.5\textwidth}
    \centering
    \includegraphics[width=5cm]{figs/figure_sample.pdf}
  \end{minipage}

\caption{$dN/dz$ for our full sample; in both, co-moving and r200 units}
\label{fig:fig2}

\end{figure}


\subsection{The incidence \mgii\ in clusters as a function of cluster mass}
\ntcomment{\\
- Define cluster mass bins; explain why we chose these particular bins.\\
- Main result of the paper: incidence as a function of cluster masses\\
- Plot incidence as a function of cluster masses, in r200 (or both in case is worth it)}



\subsection{Physical model}
\ntcomment{\\ 
- Explain our model to fit the results\\ 
- Mention whether our model consistent with the known masses of the
clusters in each mass bin?  (either result will be very
interesting) but leave full interpretation to Discussion section\\ }

\begin{figure}
 \begin{minipage}{0.5\textwidth}
    \centering
    \includegraphics[width=5cm]{figs/figure_sample.pdf}
  \end{minipage}

\caption{$dN/dz$ for our cluster-mass subsamples; in both, co-moving and r200 units; this figure should contain our model}
\label{fig:fig3}

\end{figure}


\section{Discussion}
\ntcomment{\\
- General implications of results, focusing on the main result\\
- Although we focused on the larger scales, we can speculate now on our reported results on scales < 1 Mpc (subject to large statistical uncertainties).\\ 
- Why is important to have cold gas in clusters\\
- Are the observed trends consistent with expectations?\\
}

\subsection{Comparison to previous work}
\ntcomment{\\
- Lopez+2008\\
- Zhu+14\\
- Gauthier+14\\
- etc. \\
- emphasize why our experiment is different (we focus on the most massive haloes in the universe and we know the masses of the clusters very well)\\
- focus on agreements and disagreements\\
}

\subsection{Future prospects [??]}
\ntcomment{\\
- What can be done to extend this work?\\
- Is it worth to aim for less massive groups? Is there evidence for a different behaviour in these lower mass haloes?\\
- Mention work on the smaller scales that we will do ?\\
- Address different galaxy populations (star-forming, non-star-forming)\\
- Split by \mgii\ EW for a fixed cluster mass\\
- Investigate properties of clusters showing and not showing \mgii.\\
- Put limits on the covering fraction of \mgii inside galaxy clusters [this seems important, should we address this in this paper rather than later? We could to it for the strongest absorbers without worrying much about incompleteness].\\
}



\section{Summary}
\ntcomment{Brief summary}

\section*{Acknowledgments}

We thank contributors to SciPy\footnote{\url{http://www.scipy.org}},
Matplotlib\footnote{\url{http://www.matplotlib.sourceforge.net}},
Astropy\footnote{\url{http://www.astropy.org}\citep{AstropyCollaboration2013}},
and the Python programming
language\footnote{\url{http://www.python.org}}; the free and
open-source community; and the NASA Astrophysics Data
System\footnote{\url{http://adswww.harvard.edu}} for software and
services.

Add \ac{sdss} thanks.

N.H.C. acknowledges...; N.T. acknowledges...; S.L. acknowledges...

\bibliographystyle{mn2e_trunc8.bst}
\bibliography{/home/ntejos/lit/bib/IGM}

\appendix

\section{Check for systematic effects}
\ntcomment{\\ 
-Plot QSO properties as a function of cluster masses, and transverse
separation in the same way the analysis was presented.\\
-Plot properties of clusters (mass distribution, redshift distribution)
with \mgii\ versus those without ?}
\begin{figure}
 \begin{minipage}{0.5\textwidth}
    \centering
    \includegraphics[width=5cm]{figs/figure_sample.pdf}
  \end{minipage}

\caption{Properties of \acp{qso} as a function of transverse distance.}
\label{fig:fig_appendix}

\end{figure}



\bsp
\label{lastpage}
\end{document}
